\documentclass{article}
\usepackage{amsmath, amssymb, amsthm, hyperref}

\title{Homework Assignment 1}
\author{Mark}
\date{\today}

\begin{document}

\maketitle

\begin{abstract}
Note: This is my first time using LaTeX, feel free to point out areas of improvement. Furthermore, I used AI to help me with LaTeX syntax but did not use it for the math portions.
\end{abstract}

\section{Problem 1}
Let $n \in \mathbb{N}$, define the relation $R$ on $S = \mathbb{Z}$ as follows: For $a, b \in \mathbb{Z}$, we have $aRb$ if $a - b$ is divisible by $n$. Is $R$ an equivalence relation on $\mathbb{Z}$? Justify your answer.

\noindent \textbf{Solution:} 
Relation $R$ is an equivalence relation on $\mathbb{Z}$ iff the relation is reflexive, symmetric, and transitive. I will prove that $R$ is an equivalence relation (assuming $0\notin\mathbb{N}$)
\begin{proof}

1. \textbf{\textit{Reflexive:}} Assume $a \in \mathbb{Z}$, then $aRa$ is $a-a|n = 0 | n$. 0 is divisible by all numbers other than 0. $R$ is reflexive. If $0\in \mathbb{N}$, then this would not be an equivalence relation, because nothing is divisible by 0.
   
2. \textbf{\textit{Symmetry:}} Assume $aRb$, then $a-b|n \implies a-b=n*k$ for some $k \in \mathbb{Z}$. We can multiply both sides by $-1$. $-1(a-b)=-1(n*k) $ which simplifies to $b-a=n*(-k)$, which is $bRa$. Therefore $aRb$ implies $bRa$, specifically the resulting integer of $bRa$ will be the negative of $aRb$

3. \textbf{\textit{Transitivity:}} Suppose $aRb$ and $bRc$. This means $a - b = k_1n$ and $b - c = k_2n$ for some $k, l \in \mathbb{Z}$.Adding these, we get $(a-b) + (b-c) = nk_1 + nk_2 \implies a-c = (k_1+k_2)*n$, where $k_1+k_2 \in \mathbb{Z}$. Therefore, $aRb, bRc \implies aRc$ \\
   
Because $R$ is reflexive, symmetric, and transitive, $R$ is an equivalence relation on $\mathbb{Z}$, assuming $0 \notin \mathbb{N}$.
\end{proof}

\section{Problem 2}
Let $n \in \mathbb{N}$
\subsection{Part (a)}
Show that $(\mathbb{Z}\ n\mathbb{Z})$ is a monoid under the operation of multiplication. Assume $ab=a*b$, and that $(\mathbb{Z}\ n\mathbb{Z})$ is $G$

\noindent \textbf{Solution:} 
$(\mathbb{Z}\ n\mathbb{Z})$ is the set of equivalence classes $\{\overline{0}, \overline{1}, \ldots, \overline{n-1}\}$. We must prove association and that an identity element exists.

\begin{proof}
1. \textbf{\textit{Associativity:}} We must prove that $\overline{a} (\overline{b} \overline{c}) = (\overline{a} \overline{b}) \overline{c}$. By definition $\overline{a}*\overline{b}=\overline{a*b}$ Thus, we can simplify both sides. The left-hand side simplifies to $\overline{a}*\overline{b*c} = \overline{a*b*c}$, while the right-hand side simplifies to $\overline{a*b}*\overline{c} = \overline{a*b*c}$, which are equal.

Therefore, $(\mathbb{Z}\ n\mathbb{Z})$ is associative.

2. \textbf{\textit{Identity Element:}} We must prove that there exists some $1_G$ st $a*1_G=a=1_G*a \quad \forall a \in G$. Clearly, this element is $\overline{1}$, by definition $\overline{a}*\overline{1}=\overline{a*1}=\overline{a}$

Therefore, $(\mathbb{Z}\ n\mathbb{Z})$ is a monoid.
\end{proof}

\subsection{Part (b)}
Show that $\overline{x}$ belongs to the unit group of $(\mathbb{Z}\ n\mathbb{Z})$ if and only if $x$ and $n$ are coprime

\noindent \textbf{Solution:} 
\begin{proof}
1. We will show that if $x$ and $n$ are coprime, then $\overline{x}$ belongs to $(\mathbb{Z}\ n\mathbb{Z})^X$. $\overline{x}$ exists in $U_n$ iff there exists some inverse such that $\overline{x}*\overline{x^{-1}}\equiv\overline{1}$. Because \href{https://en.wikipedia.org/wiki/B%C3%A9zout%27s_identity}{Bézout's identity}, we know that some $ax+bn=1$, furthermore, because we are in the unit group, this fact can be used to say that $ax\equiv 1 (mod n)$. Therefore, in this set, $x$ has an inverse $a$ s.t. their product mod n is 1, and therefore they would (by definition) have to exist in the set
\end{proof}

\subsection{Part (c)}
List all the elements of the unit group $U(\mathbb{Z}\ 15\mathbb{Z})$. You may use results from part

\noindent \textbf{Solution:} 
Clearly 1. Because they need to be coprime, 3, 5, 6, 9, 10, and 12 are out. So, we are left with
\{1, 2, 4, 7, 8, 11, 13, 14\} their respective inverses are \{1, 8, 4, 13, 2, 11, 7, 14\}

\subsection{Part (d)}
Find the orders of $\overline{2}, \overline{4}, \overline{7}$ in $U(\mathbb{Z}\ 15\mathbb{Z})$. Justify your answer
\noindent \textbf{Solution:} 
The order of an element in a group is the lowest $n \in \mathbb{N}$ s.t. $g^n=1_G$, in our case $\overline{g}^n \equiv 1(\textrm{ mod } 15)$. I can just brute force this.
1. For $\overline{2}$, it is $2^4=16$, which mod 15 is 1. $n=4$
2. For $\overline{4}$, it is $4^2=16$. $n=2$
3. For $\overline{7}$, it is $7^4=2401$, which mod 15 is also 1 $n=4$

\section{Problem 3}
Let $G$ be the set of real numbers $(a, b) \in \mathbb{R}^2 \textrm{ with }a\neq 0$ and define \\
$$(a,b)*(c,d)=(ac, ad+b) \quad 1_G=(1,0)$$
Verify that this defines a group \\
\noindent \textbf{Solution:} 
    G is a group iff it is associative, there is an identity element $1_G$, and there is an inverse element $\forall (a,b) \in G$. Assume that $(a,b)(c,d)=(a,b)*(c,d)$
\begin{proof}

1. \textbf{\textit{Associativity:}} Manually check that $a(bc)=(ab)c$, or in our case $(a,b)((c,d)(e,f))=((a,b)(c,d))(e,f) \quad \forall (a,b),(c,d),(e,f) \in G$ \\
$$(a,b)((c,d)(e,f))=(a,b)(cd, cf+d)=(ace, a(cf+d)+b)=(ace, acf+ad+b)$$
$$((a,b)(c,d))(e,f)=(ac,ad+b)(e,f)=(ace, acf+ad+b)$$
These two equations are clearly equal, so G is associative.

2. \textbf{\textit{Identity Element:}} We need to prove that the given identity element $1_G=(1,0)$ holds all properties, specifically $(a,b)1_G=(a,b)=1_G(a,b) \quad \forall a,b \in G$. Again, we can manually check that this is the case. Assume $(a,b) \in G$
$$(a,b)(1,0)=(a,0+b)\quad (1,0)(a,b)=(a, b+0)$$ 
Clearly, this holds, therefore $G$ has an identity element $1_G=(1,0)$

3. \textbf{\textit{Inverse Elements:}} We now must prove that $\forall (a,b) \in G$ there is some $(c, d) \textrm{st} (a,b)(c,d)=1_G$. Suppose $(a,b) \in G$ and there is some $(c,d) \in G$. $(a,b)(c,d)=(ac, ad+b)$ for this to be the identity, we get two equations. $ac=1 \quad \textrm{and} \quad ad+b=0$, which solve to $c=\frac{1}{a} \quad \textrm{and} \quad d=-\frac{b}{a}$. Thus for every $(a,b) \quad a\neq 0$ has an inverse, specifically $(\frac{1}{a},-\frac{b}{a}$ \\
Because G is associative, has an identity element that follows all the rules, and has an inverse element for all sets, G is a group.
\end{proof}

\section{Problem 4}
For $\theta \in (0, 2\pi)$, define the rotation map $R$ and the vertical mirror symmetric map S on the plane $\mathbb{R}^2$ as follows:
$$R(x, y) = (x \cos \theta - y \sin \theta, x \sin \theta + y \cos \theta), \quad S(x,y) = (x, -y)$$
for $(x, y) \in \mathbb{R}^2$. Show that $RSR=S$.

\noindent \textbf{Solution:} 
We can simply manually check that this is the case. Take the set $T$. $RSR$ on $T$ would be to apply the map $R$, then $S$, then $R$ again onto $T$, and show that that is the same as just applying $S$ to $T$.
\begin{proof} Let $(x, y) \in \mathbb{R}^2$. $R \textrm{ on } \mathbb{R}^2 = (x \cos \theta - y \sin \theta, x \sin \theta + y \cos \theta)$, next, $RS \textrm{ on } \mathbb{R}^2 = (x \cos \theta - y \sin \theta, -x \sin \theta - y \cos \theta)$, next (grossly), 
$$RSR \textrm{ on } \mathbb{R}^2 = ((x \cos \theta - y \sin \theta)\cos \theta - (- x \sin \theta - y \cos \theta) \sin \theta), (x \cos \theta - y \sin \theta)\sin \theta + (- x \sin \theta - y \cos \theta)\cos \theta$$. 
This can be simplified into $x \cos^2 \theta + x \sin^2 \theta, -(y \sin^2 \theta + y \cos^2 \theta)$. Which, using trig identities, is just $(x, -y)$, which is the same result as applying $S$ onto $\mathbb{R}^2$. Thus, applying $RSR:\mathbb{R}^2 \textrm{ is the same as applying } S:\mathbb{R}^2$

\end{proof}

\section{Problem 5}
Show that in a group $(G,*)$, the equations $a*x=b$ and $y*a=b$ are solvable for any $a,b \in G$.

\noindent \textbf{Solution:} 
\begin{proof}
    Suppose $a,b \in G$ and they have inverses $a^{-1}, b^{-1} \in G$.
    $a*x=b \rightarrow a^{-1} * a * x = a^{-1}*b$. By identity, $x=a^{-1}*b$. From this same logic $y*a=b \rightarrow y=b*a^{-1}$. This is solvable because $a, b, a^{-1}, b^{-1} \in G$, through our assumption and inverse rules.
\end{proof}

\section{Problem 6}
For an arbitrary group $G$, the center of $G$, denoted $C(G)$, is a subset of $G$ consisting of all elements which commute with every element of $G$, that is,
$$ C(G) := \{ g \in G \mid gx = xg \ \text{for all} \ x \in G \}. $$
For any group $G$, prove that $C(G)$ is a subgroup of $G$.

\noindent \textbf{Solution:} 
For $C(G)$ to be a subgroup of $G$, then $1_G \in C(G)$, $h, k \in C(G) \implies hk \in C(G)$, the inverse exists $\forall h \in C(G)$
\begin{proof}
1. \textbf{\textit{Identity Element:}} By definition, $1_G * x = x * 1_G \forall x \in G$, thus, the identity element must be in $C(G)$

2. \textbf{\textit{Closed Under Products:}} Assume there exists some $h, k \in C(G)$. This implies that $hx=xh \quad \textrm{and} \quad kx=xk$. $khx=kxh$, by our equation $kx=xk$, we can pass $hx$ as $x$, meaning that $k(xh)=(xh)k$ using our first equation, $(kh)x = x(kh)$ is proved directly, meaning that $\forall k, h \in C(G), kh \in C(G)$

3. \textbf{\textit{Inverse exists:}} Assume there is $h \in C(G)$. By definition $hx = xh \forall x \in C(G)$ we can multiply both sides by the inverse of h $h^{-1}hxh^{-1}=h^{-1}xhh^{-1} \implies xh^{-1}=h^{-1}x$. So, $hx = xh $ implies $xh^{-1}=h^{-1}x$
\end{proof}

\section{Problem 7}
Let $G$ be a group, and assume that $X^2=1$ for all $x \in G$. Show that $G$ is abelian.

\noindent \textbf{Solution:} Group $G$ is abelian iff $ab=ba \quad a,b \in G$. We can prove this directly.
\begin{proof} Assume some $x, y \in G$. We will prove that $xy=yx$
$x^2=1$, we can multiply both sides by the inverse of x, $x^{-1}xx=x^{-1}1 \rightarrow x=x^{-1} \forall x \in G$. Thus, $xy=x^{-1}y^{-1}$. The right-hand side can be simplified into $(yx)^{-1}$. Because the product of x and y is also in the group, we can use our first equation and get $(yx)^{-1}=(yx)$, which can be substituted as $xy=yx \forall x, y \in G$.

Therefore, G is abelian.
\end{proof}

\section{Problem 8}
Let $G$ be a group. Show that

\subsection{(Part (a))}
For any $x, y \in G$, we have $o(x)=o(y^{-1}xy$
\noindent \textbf{Solution:} 
By definition, $o(x)=$ the smallest $n \in \mathbb{N}$ s.t. $x^n=1_G$
\begin{proof}
Assume $x^n = 1_G$ and $(yxy^{-1})^m = 1_G$, and that $m \neq n$. We can expand the right side to be $1_G = yxy^{-1} yxy^{-1} \ldots yxy^{-1}$ (with $m$ factors), and, using associativity and the identity property of inverses, we can simplify this to $1_G = y x^m y^{-1}$. This leads to the conclusion that $1_G = x^m = x^n$.

Therefore, $m$ must equal $n$.
\end{proof}
For any $a, b \in G$, we have $o(ab)=o(ba)$
\subsection{(Part (b))}
\noindent \textbf{Solution:} 
The solution here is pretty simple and just follows the definition and what we previously proved.
\begin{proof}
We know that $o(b)=o(a^{-1}ba)$. Now, we can simply take $b=ab$, subbing this in gives us $o(ab)=o(a^{-1}aba)$, which because of identity, is just $o(ab)=o(ba)$
\end{proof}

\section{Problem 9}
Let $G$ be a group and $H$ be a non-empty subset of $G$. Show that $H$ is a subgroup of $G$ if and only if every $h \in H$ is invertible in $G$, and $h_1^{-1}h_2 \in H$ for all $h_1, h_2 \in H$

\noindent \textbf{Solution:} 
This can be proved essentially through definition. A subgroup must follow the following 3 properties. 1) $1_G \in H$, 2) $h, k \in H \implies hk \in H$, 3) $\forall h \in H \exists h^{-1}$
\begin{proof}
Assume $h_1, h_2, h_1^{-1} \in H$.\\
1. \textbf{\textit{Identity Element:}} Because we are assuming that $\forall h \in H \exists h^{-1}$, by definition, their operation must be the identity element, and because those exact same elements exist in $G$, that identity element is also that of $G$

2. \textbf{\textit{Closed Under Product:}} We can use some funny logic here. Because we are assuming that $h_1^{-1}h_2 \in H$ and that $h_1 \implies h_1^{-1}$, that means for every h in H, we know that the operation of it and every other element must exist, if we imagine h as the inverse of it's inverse.

3. \textbf{\textit{Closed Under Product:}} This is assumed

\end{proof}

\end{document}
